\documentclass{article}
\usepackage{amsmath, algorithm, algpseudocode, hyperref, amssymb}

\begin{document}

\title{026}
\author{VSh}

\maketitle

\section{Tags}
\begin{itemize}
    \item Totient rule;
    \item Euler's totient function;
    \item Primitive root;
    \item Multiplicative order;
\end{itemize}

\section{\href{https://projecteuler.net/problem=26}{Problem}}
Find the value of $d < 1000$ for which $1/d$ contains the longest recurring cycle 
in its decimal fraction part.

\section{Definitions}
\subsection{\href{https://en.wikipedia.org/wiki/Euler's_totient_function}{Euler's totient function}}
In number theory, \textbf{Euler's totient function} counts the positive integers up to a 
given integer $n$ that are relatively prime to $n$.  In other words, it is the 
number of integers $k$ in the range $1 \le k \le n$ for which the greatest common 
divisor $gcd(n, k)$ is equal to 1.

\subsection{\href{https://en.wikipedia.org/wiki/Multiplicative_order}{Multiplicative order}}
In number theory, given an integer $a$ and a positive integer $n$ with $gcd(a,n) = 1$, 
the multiplicative order of a modulo $n$ is the smallest positive integer $k$ with
\begin{equation*}
    a^k \equiv 1 \pmod{n}
\end{equation*}


\section{Calculation}
\subsection{\href{https://en.wikipedia.org/wiki/Repeating_decimal\#Totient_rule}{Totient rule}} \label{totient_rule}
For an arbitrary integer $n$ the length  $\lambda (n)$ of the repetend of $1/n$ 
divides $\phi (n)$, where $\phi$  is the totient function. 
The length is equal to $\phi (n)$ if and only if 10 is a primitive root modulo $n$.

\subsection{Totient function upper bound} \label{totient_function_upper_bound}
\begin{align*}
        \phi(n) \le n - 1 & \\
        \phi(n) = n - 1  & \iff n \text{ is prime} \\
        \phi(n) < n - 1  & \iff n \text{ is not prime}
\end{align*}

According to  \ref{totient_rule} and \ref{totient_function_upper_bound} it is a 
restriction to look only trough the prime numbers from the up limit that has 10 
as a primitive roots.

\subsection{\href{https://en.wikipedia.org/wiki/Primitive_root_modulo_n\#Finding_primitive_roots}{Finding primitive roots}}
No simple general formula to compute primitive roots modulo $n$ is known. 
There are however methods to locate a primitive root that are faster than simply 
trying out all candidates. If the multiplicative order of a number $m$ modulo $n$ 
is equal to $\phi(n)$ (the order of $\mathbb{Z}_n^\times$), then it is a primitive root. 
In fact the converse is true: If $m$ is a primitive root modulo $n$, then the 
multiplicative order of $m$ is $\phi(n)$. We can use this to test for primitive roots.

First, compute $\phi(n)$. Then determine the different prime factors of $\phi(n)$:
$p_1,p_2,...,p_k$. For particular element $m$ compute
\begin{equation*}
    m^{\frac{\phi(n)}{p_i}} \pmod{n} \qquad \text{for} \; i = 1,...,k
\end{equation*}
if all $k$ results are different from $1$ then $m$ is a primitive root.
\newpage
\section{Implementation}

\begin{algorithm}
    \caption{isPrimitiveRoot}\label{alg:isPrimitiveRoot}
    \begin{algorithmic}[1]
        \State \Comment {checks if m is primitive root modulo n}
        \Procedure{isPrimitiveRoot}{m, n} 
        \Require $m$ and $n$ are coprime
        \State $eulerphi \gets$ \Call{EulerPhi}{$n$}
        \State $factors \gets$ \Call{factorise}{$eulerphi$}
        \ForAll {$factor in factors$}
            \State $power \gets eulerphi / factor$ 
            \State $t \gets$ \Call{vs::modular::power}{$m, power, n$}
            \If{$t == 1$}
                \State \textbf{return} $False$
            \EndIf
        \EndFor
        \State \textbf{return} $True$
        \EndProcedure
    \end{algorithmic}
\end{algorithm}

\begin{algorithm}
    \caption{pe026} \label{alg:pe026}
    \begin{algorithmic}[1]
        \State $n \gets 1000$
        \State $primes \gets [true, .., true]$ \Comment {n times}
        \State \Call{vs::primes::init}{primes} \Comment {Sieve for primes}
        \Repeat \Comment {Starts from highest n}
            \State $t \gets$ \Call{isPrimitiveRoot}{$10, n$}
            \State $n \gets n - 1$
            \State \Comment {until n is prime and has 10 as a primitive root}
        \Until{  $(primes[n] \land t) \lor (n < 7)$  }
    \end{algorithmic}
\end{algorithm}


\end{document}