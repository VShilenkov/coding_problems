\documentclass{article}
\usepackage{amsmath, algorithm, algpseudocode, hyperref}

\begin{document}

\title{023}
\author{VSh}

\maketitle

\section{Tags}
\begin{itemize}
    \item abundant numbers;
    \item proper divisors;
    \item function of divisors;
    \item sum-of-divisors function;
    \item aliquot sum;
\end{itemize}

\section{Problem}
Find the sum of all the positive integers which cannot be written as the sum of 
two abundant numbers.

\section{Definitions}
\subsection{Function of divisors}
The \textbf{sum of positive divisors function $\sigma_{x}(n)$}, for a real or complex number $x$,
is defined as the sum of the $x$-th powers of the positive divisors of $n$.
\begin{equation*}
    \sigma_{x}(n) = \sum_{d \mid n}{d^{x}}
\end{equation*}
where $d \mid n$ means $d$ divides $n$.

\subsection{Sum-of-divisors function}
The divisors function $\sigma_{x}(n)$ for $x=1$ called \textbf{sum-of-divisors function}.
\begin{equation*}
    \sigma_{1}(n) = \sum_{d \mid n}{d}
\end{equation*}

\subsection{Proper divisor}
A positive divisor of $n$ which is different from $n$ is called \textbf{a proper divisor} 
or \textbf{an aliquot part} of $n$.

\subsection{Aliquot sum}
The \textbf{aliquot sum $s(n)$} is the sum of proper divisors $n$.
\begin{equation*}
    s(n) = \sigma_{1}(n) - n
\end{equation*}

\subsection{Abundant number}
\textbf{An abundant number} or \textbf{excessive number} is a number for which the sum of its 
proper divisors is greater than the number itself.

To test if a particular number is abundant:
\begin{equation*}
    s(n) > n
\end{equation*}
And according to the definition of aliquot sum, can be rewritten as:
\begin{equation*}
    \sigma_{1}(n) > 2n
\end{equation*}

\section{Calculation}
\subsection{Aproach}
As it was told [\href{https://projecteuler.net/problem=23}{023}], all integers greater than 
28123 can be written as the sum of two abundant numbers. The main idea is to find all the integers 
form 24 to this upper bound that can be represented and find sum of all other.
It is only needed to check if a particular number abundant. Sum-of-divisors function can 
be utilised to test it. Then go throgh all abundant and build all possible sums of them.
All other numbers that cannot be represented is the needed sum.

\subsection{Divisors function}
Factorisation of $n$
\begin{equation}
    n = \prod_{i=1}^{r}{p_{i}^{\alpha_{i}}}
\end{equation}
Where:
\begin{itemize}
    \item $r = \omega(n)$ - the number of distinct prime numbers of $n$
    \item $p_{i}$ - $i$-th prime factor of $n$
    \item $\alpha_{i}$ - the highest power of prime $p_{i}$ by which $n$ is divisible
\end{itemize}
\begin{equation}
    \sigma_{x}(n) = \prod_{i=1}^{r}{ \sum_{j=0}^{\alpha_{i}}{p_{i}^{jx}}} = 
    \prod_{i=1}^{r}{(1 + p_{i}^{x} + p_{i}^{2x} + .. + + p_{i}^{\alpha_{i}x})}
\end{equation}
\begin{equation} \label{eq:3}
    \sigma_{x}(n) = \prod_{i=1}^{r}{  \frac{p_{i}^{(\alpha_{i}+1)x } - 1}{p_{i}^{x} - 1} }
\end{equation}
According to \eqref{eq:3} sum-of-divisors function can be calculated
\begin{equation} 
    \sigma_{1}(n) = \prod_{i=1}^{r}{  \frac{p_{i}^{(\alpha_{i}+1) } - 1}{p_{i} - 1} }
\end{equation}

\section{Implementation}
Modified Eratosthen's sieve will be used to calculated sum-of-divisors function.

\begin{algorithm}
\caption{Sum-of-divisors algorithm}\label{alg:sigma_1}
\begin{algorithmic}[1]
\State $sigma \gets [1 .. 1]$ \Comment{N times}
\State $number \gets [1 .. N]$
\For{$i \gets 2, N$}
    \If{$number_i \ne 1$} \Comment{Invariant: number[i] prime or 1}
        \State $prime \gets number_i$
        \For{$j \gets prime, N$} \Comment{For all numbers devisible by prime}
            \State $nominator \gets 1$
            \State $denominator \gets prime - 1$
            \While {$number_j \bmod prime = 0 $} \Comment{Highest power of prime}
                \State $nominator \gets nominator \cdot prime$
                \State $number_j \gets number_j / prime$
            \EndWhile
            \State $nominator \gets nominator \cdot prime - 1$
            \State $sigma_i \gets sigma_i \cdot nominator / denominator$
        \State $j \gets j + prime$
        \EndFor
    \EndIf
    \State $i \gets i + 1$
\EndFor
\end{algorithmic}
\end{algorithm}

\newpage
Mark all abundant numbers 

\begin{algorithm}
    \caption{Abundant test}\label{alg:abundant}
    \begin{algorithmic}[1]
        \State $abundant \gets [\texttt{false}..\texttt{false}]$
        \For{$i \gets 2, N$}
            \If{$sigma_i > 2i$ }
                \State $abundant \gets \texttt{true}$
            \EndIf
            \State $i \gets i + 1$
        \EndFor
    \end{algorithmic}
\end{algorithm}

Build all possible sums of pair of abundant numbers

\begin{algorithm}
    \caption{Sums of abundant}
    \begin{algorithmic}[1]
        \State $represented \gets [\texttt{false} .. \texttt{false}]$
        \For{$i \gets 1, N$}
            \State $i \gets i + 1$
            \If{$abundant_i$}
                \For{$j \gets i, N - i$}
                    \If{$abundant_i$}
                        \State $represented_{i + j} \gets \texttt{true}$
                    \EndIf
                    \State $j \gets j + 1$
                \EndFor
            \EndIf
        \EndFor
    \end{algorithmic}
\end{algorithm}

Accumulate all the integers that cannot be the sum of pair of abundant numbers

\begin{algorithm}
    \caption{Sum of not representable as sum of two abundant}
    \begin{algorithmic}[1]
        \State $total \gets 0$
        \For{$i \gets 1, N$}
            \If{$\neg represented_i$}
                \State $total \gets total + i$
            \EndIf
            \State $i \gets i + 1$
        \EndFor
    \end{algorithmic}
\end{algorithm}

\end{document}