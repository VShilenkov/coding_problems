\documentclass{article}
\usepackage{amsmath, algorithm, algpseudocode, hyperref}

\begin{document}

\title{024}
\author{VSh}

\maketitle

\section{Tags}
\begin{itemize}
    \item permutation;
    \item lexicographic permutation;
    \item factorial representaion;
\end{itemize}

\section{Problem}
Find the millionth lexicographic permutation of the digits ${0, 1, 2, 3, 4, 5, 6, 7, 8, 9}$.

\section{Definitions}
\subsection{Permutation}
[\href{https://en.wikipedia.org/wiki/Permutation}{wiki:Premutation}]\newline
\textbf{Permutation} is the act of \textbf{arranging} the members of a set into
a sequence or order, or, if the set is already ordered, \textbf{rearranging}
(reordering) its elements — a process called \textbf{permuting}.

\subsection{Lexicographic permutation}
\textbf{Lexicographic permutation} - the way of permuting the set, where two
consequtive permutations is in lexicographic order.

\subsection{Factorial rerepresentaion (Factorial number system)}
[\href{https://en.wikipedia.org/wiki/Factorial_number_system}{wiki:Factorial number system}]\newline
In combinatorics, the \textbf{factorial number system}, also called \textbf{factoradic},
is a mixed radix numeral system adapted to numbering permutations. It is also
called \textbf{factorial base}, although factorials do not function as base,
but as place value of digits. By converting a number less than $n!$ to factorial
representation, one obtains a sequence of $n$ digits that can be converted to a
permutation of $n$ in a straightforward way, either using them as Lehmer code
or as inversion table representation; in the former case the resulting map from
integers to permutations of $n$ lists them in lexicographical order.

\section{Calculation}
\subsection{Approach}
Using factorial base number system it is possible to find $n_{th}$ lexicographic
permutation, instead of finding all of them and looking for $n_{th}$.

\subsection{Factorial number system}
A factorial number system uses factorial values instead of powers of numbers
to denote place-values (or base). \newline

\begin{tabular}{ | l | c | c | c | c | c | c | c | c |}
    \hline
    \textbf{Radix}                  & 8    & 7   & 6   & 5  & 4  & 3  & 2  & 1  \\ \hline
    \textbf{Place value}            & 7!   & 6!  & 5!  & 4! & 3! & 2! & 1! & 0! \\ \hline
    \textbf{Place value in decimal} & 5040 & 720 & 120 & 24 & 6  & 2  & 1  & 1  \\ \hline
    \textbf{Highest digit allowed}  & 7    & 6   & 5   & 4  & 3  & 2  & 1  & 0  \\ \hline
\end{tabular}
\newline \newline

For example:
\begin{align*}
    \hline
    00_{10} & \mapsto & 0 \cdot 3! + 0 \cdot 2! + 0 \cdot 1! + 0 \cdot 0! \mapsto 0000_{!} \\
    01_{10} & \mapsto & 0 \cdot 3! + 0 \cdot 2! + 1 \cdot 1! + 0 \cdot 0! \mapsto 0010_{!} \\
    02_{10} & \mapsto & 0 \cdot 3! + 1 \cdot 2! + 0 \cdot 1! + 0 \cdot 0! \mapsto 0100_{!} \\
    03_{10} & \mapsto & 0 \cdot 3! + 1 \cdot 2! + 1 \cdot 1! + 0 \cdot 0! \mapsto 0110_{!} \\
    04_{10} & \mapsto & 0 \cdot 3! + 2 \cdot 2! + 0 \cdot 1! + 0 \cdot 0! \mapsto 0200_{!} \\
    05_{10} & \mapsto & 0 \cdot 3! + 2 \cdot 2! + 1 \cdot 1! + 0 \cdot 0! \mapsto 0210_{!} \\
    06_{10} & \mapsto & 1 \cdot 3! + 0 \cdot 2! + 0 \cdot 1! + 0 \cdot 0! \mapsto 1000_{!} \\
    07_{10} & \mapsto & 1 \cdot 3! + 0 \cdot 2! + 1 \cdot 1! + 0 \cdot 0! \mapsto 1010_{!} \\
    08_{10} & \mapsto & 1 \cdot 3! + 1 \cdot 2! + 0 \cdot 1! + 0 \cdot 0! \mapsto 1100_{!} \\
    09_{10} & \mapsto & 1 \cdot 3! + 1 \cdot 2! + 1 \cdot 1! + 0 \cdot 0! \mapsto 1110_{!} \\
    10_{10} & \mapsto & 1 \cdot 3! + 2 \cdot 2! + 0 \cdot 1! + 0 \cdot 0! \mapsto 1200_{!} \\
   \hline
\end{align*}

\newpage
\subsection{Lexicographic permutation}

Example of permutations of set $\{a, b, c, d\}$

\begin{align*}
    \hline
    00 \quad abcd && 06 \quad bacd && 12 \quad cabd && 18 \quad dabc \\
    01 \quad abdc && 07 \quad badc && 13 \quad cadb && 19 \quad dacb \\
    02 \quad acbd && 08 \quad bcad && 14 \quad cbad && 20 \quad dbac \\
    03 \quad acdb && 09 \quad bcda && 15 \quad cbda && 21 \quad dbca \\
    04 \quad adbc && 10 \quad bdac && 16 \quad cdab && 22 \quad dcab \\
    05 \quad adcb && 11 \quad bdca && 17 \quad cdba && 23 \quad dcba \\
    \hline
\end{align*}
\newline

The first letter changes after every $6_{th} (3!)$ permutation. The second one -
after $2 (2!)$, third one and forth one changes every time ($1!$ and $0!$).
\newline\newline
The algorithm of $n_{th}$ lexicographical permutation looks like:
\begin{enumerate}
    \item represent $n$ in factoradic;
    \item sort lexicographically input set and consider indexes from 0;
    \item for each digit in factoriadic representaion:
    \begin{enumerate}
            \item append the symbol from input set by this number to the output
            \item remove from input
    \end{enumerate}
\end{enumerate} 

\paragraph{example}
Find $14_{th}$ lexicographical permutation of $\{a,b,c,d\}$ set.
\begin{equation*}
    14_{10} = 2 \cdot 3! + 1 \cdot 2! + 0 \cdot 1! + 0 \cdot 0! = 2100_{!}
\end{equation*}
Initial state: \qquad
\begin{tabular}{ l | c | c }
    \hline
    \quad       & input         & output    \\
    string      & \textbf{abcd} & \textbf{} \\
    indexes     &  0123         & \quad     \\
    factoriadic & \quad         & 2100      \\
\end{tabular}
\newline\newline
First digit in factoradic is 2. The item from the set with index 2 is \textbf{c}
\newline
Iteration: 1 \qquad
\begin{tabular}{ l | c | c }
    \hline
    \quad       & input         & output    \\
    string      & \textbf{abd} & \textbf{c} \\
    indexes     &  012         & \quad      \\
    factoriadic & \quad        & 100        \\
\end{tabular}
\newline\newline
Next digit - 1. Item - \textbf{b} \newline
Iteration: 2 \qquad
\begin{tabular}{ l | c | c }
    \hline
    \quad       & input       & output      \\
    string      & \textbf{ad} & \textbf{cb} \\
    indexes     &  01         & \quad       \\
    factoriadic & \quad       & 00          \\
\end{tabular}
\newline\newline

Next digit - 0. Item - \textbf{a} \newline
Iteration: 3 \qquad
\begin{tabular}{ l | c | c }
    \hline
    \quad       & input      & output       \\
    string      & \textbf{d} & \textbf{cba} \\
    indexes     &  0         & \quad        \\
    factoriadic & \quad      & 0            \\
\end{tabular}
\newline\newline

Last digit - 0. Item - \textbf{d} \newline
Iteration: 4 \qquad
\begin{tabular}{ l | c | c }
    \hline
    \quad       & input      & output       \\
    string      & \textbf{} & \textbf{cbad} \\
\end{tabular}
\newline\newline

\section{Implementation}

\begin{algorithm}
    \caption{Factoriadic representaion}\label{alg:factoriadic}
    \begin{algorithmic}[1]
        \Require $number$ \Comment{takes number to transform in factoriadic}
        \State $base \gets 10$
        \State $representation \gets [0..0]$ \Comment{10 times ($base$)}
        \For{$i \gets 2, i \le base$}
            \State $representation_{base-i} = number \bmod i$
            \State $number \gets number \div i$
            \State $i \gets i + 1$
        \EndFor
        \State \Return $representation$
    \end{algorithmic}
\end{algorithm}

\begin{algorithm}
    \caption{lexicographical permutation}\label{alg:permutation}
    \begin{algorithmic}[1]
        \Require $representaion$ \Comment{permitation number in factoriadic}
        \State $input \gets [0..base]$
        \State $output \gets []$ \Comment{empty string}
        \For{$i \gets 0, i < base$}
            \State $index \gets representaion_{i} $
            \State $p \gets input_{index}$
            \State $output.append(p)$
            \State $input.erase(index)$
            \State $i \gets i + 1$
        \EndFor
        \State \Return $output$
    \end{algorithmic}
\end{algorithm}


\end{document}