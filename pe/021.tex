\documentclass{article}
\usepackage{amsmath, algorithm, algpseudocode}

\begin{document}

\title{021}
\author{VSh}

\maketitle

\section{Tags}
\begin{itemize}
    \item amicable numbers;
    \item proper divisors;
    \item function of divisors;
    \item sum-of-divisors function;
    \item aliquot sum;
\end{itemize}

\section{Problem}
Evaluate the sum of all the amicable numbers under 10000.

\section{Definitions}
\subsection{Function of divisors}
The \textbf{sum of positive divisors function $\sigma_{x}(n)$}, for a real or complex number $x$,
is defined as the sum of the $x$-th powers of the positive divisors of $n$.
\begin{equation*}
    \sigma_{x}(n) = \sum_{d \mid n}{d^{x}}
\end{equation*}
where $d \mid n$ means $d$ divides $n$.

\subsection{Sum-of-divisors function}
The divisors function $\sigma_{x}(n)$ for $x=1$ called \textbf{sum-of-divisors function}.
\begin{equation*}
    \sigma_{1}(n) = \sum_{d \mid n}{d}
\end{equation*}

\subsection{Proper divisor}
A positive divisor of $n$ which is different from $n$ is called \textbf{a proper divisor} 
or \textbf{an aliquot part} of $n$.

\subsection{Aliquot sum}
The \textbf{aliquot sum $s(n)$} is the sum of proper divisors $n$.
\begin{equation*}
    s(n) = \sigma_{1}(n) - n
\end{equation*}

\subsection{Amicable numbers}
\textbf{Amicable numbers} are two different numbers so related that the sum of the proper divisors of each is equal to the other number.
\begin{equation*}
    \begin{split}
        s(n) = m \\
        s(m) = n
    \end{split}
\end{equation*}
And to test if a number is one of amicable pair:
\begin{equation*}
        n = s(s(n))
\end{equation*}

\section{Calculation}
To find out if the number is one of amicable pair, needs to be calculated the aliquot sum of this number,
and for that need to know the sum-of-divisors function for that number.

\subsection{Divisors function}
Factorisation of $n$
\begin{equation}
    n = \prod_{i=1}^{r}{p_{i}^{\alpha_{i}}}
\end{equation}
Where:
\begin{itemize}
    \item $r = \omega(n)$ - the number of distinct prime numbers of $n$
    \item $p_{i}$ - $i$-th prime factor of $n$
    \item $\alpha_{i}$ - the highest power of prime $p_{i}$ by which $n$ is divisible
\end{itemize}
\begin{equation}
    \sigma_{x}(n) = \prod_{i=1}^{r}{ \sum_{j=0}^{\alpha_{i}}{p_{i}^{jx}}} = 
    \prod_{i=1}^{r}{(1 + p_{i}^{x} + p_{i}^{2x} + .. + + p_{i}^{\alpha_{i}x})}
\end{equation}
\begin{equation} \label{eq:3}
    \sigma_{x}(n) = \prod_{i=1}^{r}{  \frac{p_{i}^{(\alpha_{i}+1)x } - 1}{p_{i}^{x} - 1} }
\end{equation}
According to \eqref{eq:3} sum-of-divisors function can be calculated
\begin{equation} 
    \sigma_{1}(n) = \prod_{i=1}^{r}{  \frac{p_{i}^{(\alpha_{i}+1) } - 1}{p_{i} - 1} }
\end{equation}

\section{Implementation}
Modified Eratosthen's sieve will be used to calculated sum-of-divisors function.
\begin{algorithm}
    \caption{Sum-of-divisors algorithm}\label{alg:sigma_1}
    \begin{algorithmic}[1]
        \State $sigma \gets [1 .. 1]$ \Comment{N times}
        \State $number \gets [1 .. N]$
        \For{$i \gets 2, N$}
            \If{$number_i \ne 1$} \Comment{Invariant: number[i] prime or 1}
                \State $prime \gets number_i$
                \For{$j \gets prime, N$} \Comment{For all numbers devisible by prime}
                    \State $nominator \gets 1$
                    \State $denominator \gets prime - 1$
                    \While {$number_j \bmod prime = 0 $} \Comment{Highest power of prime}
                        \State $nominator \gets nominator \cdot prime$
                        \State $number_j \gets number_j / prime$
                    \EndWhile
                    \State $nominator \gets nominator \cdot prime - 1$
                    \State $sigma_i \gets sigma_i \cdot nominator / denominator$
                \State $j \gets j + prime$
                \EndFor
            \EndIf
            \State $i \gets i + 1$
        \EndFor
    \end{algorithmic}
\end{algorithm}

Get aliqout sums for all the numbers:
\begin{algorithm}
    \caption{Aliquot sum} \label{alg:aliqout}
    \begin{algorithmic}[1]
        \For{$i \gets 2, N$}
            \State $sigma_i \gets sigma_i - i$
            \State $i \gets i + 1$
        \EndFor
    \end{algorithmic}
\end{algorithm}
\newpage
Accumulate all the amicable numbers
\begin{algorithm}
    \caption{Amicable sum} \label{alg:amicable}
    \begin{algorithmic}[1]
        \State $sum \gets 0$
        \For{$i \gets 2, N$}
            \If{($i \ne sigma_i) \& (sigma_i \le N) \& (i = sigma_{sigma_i}) $}
            \State $sum \gets sum + i$
            \EndIf
            \State $i \gets i + 1$
        \EndFor
    \end{algorithmic}
\end{algorithm}

\end{document}