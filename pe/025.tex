\documentclass{article}
\usepackage{amsmath, hyperref}

\begin{document}

\title{025}
\author{VSh}

\maketitle

\section{Tags}
\begin{itemize}
    \item Fibonacci numbers;
    \item logarithm;
    \item number of digits;
\end{itemize}

\section{Problem}
Find th index of the first term in the Fibonacci sequence that contains 1000 digits.
\newline
[\href{https://projecteuler.net/problem=25}{problem}]

\section{Definitions}
\subsection{Fibonacci numbers}
[\href{https://en.wikipedia.org/wiki/Fibonacci_number}{wiki:Fibonacci numbers}] \newline
The \textbf{Fibonacci numbers}, commonly denoted $F_n$ form a sequence, called 
the \textbf{Fibonacci sequence}, such that each number is the sum of the two 
preceding ones, starting from 0 and 1.
\begin{align*}
    F_0 = & 0 \\
    F_1 = & 1 \\
    F_n = & F_{n-1} + F_{n-2}, n > 1
\end{align*}

\subsubsection{Computation by rounding}
\begin{equation*}
    F_n = \bigg[  \frac{\varphi^{n}}{\sqrt{5}}  \bigg] 
    = \bigg[  \frac{\big( \frac{1 + \sqrt{5}}{2} \big)^{n}}{\sqrt{5}}  \bigg]
\end{equation*}

\subsection{Logarithm}
[\href{https://en.wikipedia.org/wiki/Logarithm}{wiki:Logarithm}] \newline
The \textbf{logarithm} is the inverse function to exponentiation.
\begin{align*}
    \log_{b}{(x)} = y \iff b^{y} = x
\end{align*}

\subsection{Number of digits}
According to the definition of logarithm, the number of digits in representing of
the particular number $x$ in numerical system with base $b$ is the smallest integer
strictly bigger than $\log_{b}{(x)}$.

\section{Calculation}
Let $N(x)$ - function of the number of digits in representation of $x$.
\begin{equation*}
    N(x) > \lg{(x)}
\end{equation*}

Then

\begin{align*}
    N(F_n) & > \lg{\bigg[  \frac{\big( \frac{1 + \sqrt{5}}{2} \big)^{n}}{\sqrt{5}}  \bigg]}  > \\
    & >  \bigg[  n \cdot \lg{ \frac{1 + \sqrt{5}}{2} } - \frac{\lg{5}}{2}  \bigg]
\end{align*}

To find \textbf{first} Fibonacci number with 1000 digits solve inequality:
\begin{align*}
\bigg[  n \cdot \lg{ \frac{1 + \sqrt{5}}{2} } - \frac{\lg{5}}{2}  \bigg] > & 999 \\
n \cdot \lg{ \frac{1 + \sqrt{5}}{2} } > & 999 + \frac{\lg{5}}{2} \\
n > & \frac {999 + \frac{\lg{5}}{2}} {\lg{ \frac{1 + \sqrt{5}}{2} }} \\
n > & \frac {1992 + \lg{5}}{ 2 \cdot ( \lg{ (1 + \sqrt{5})} - \lg{2}) } 
\end{align*}

Solving this inequality in integers: $n = 4782$

\end{document}